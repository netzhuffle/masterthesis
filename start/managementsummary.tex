% Das Management Summary richtet sich in der Praxis an die "Chefs des Chefs",
% d.h. an die Vorgesetzten des Auftraggebers (diese sind in der Regel keine
% Fachspezialisten).
% 
% Die Sprache soll knapp, klar und stark untergliedert sein.
% Zu verwenden ist folgenden Gliederung:
% - Ausgangslage
% - Vorgehen, Technologien
% - Ergebnisse
% - Ausblick (optional)

\chapter*{Management Summary}\addcontentsline{toc}{chapter}{Management Summary}

\section*{Starting Position}\addcontentsline{toc}{section}{Starting Position}

Board and card games are not only played on tables, but also with increasing
popularity on electronic devices. This often requires every player downloading
and installing an app or web browser extension, which is a significant amount of
work needed to be able to start playing or just showing somebody a new board
game over the internet.

\section*{Approach, Technologies}\addcontentsline{toc}{section}{Approach, Technologies}

The goal of this project is to make it easier for software developers to create
board and card game implementations that are as easy to get started with for
players as is visiting a website. Players visit the game's web address and share them
with the other players to invite them to the game. Everything takes place in the
web browser using modern web technologies without needing installations or
configurations for the players. The web browsers connect and communicate to each
other enabling people to easily play board and card games over the internet with
each other.

To achieve this goal, this project creates a library for software developers
which manages the connection and communication process. The using software
developer only needs to build the board game implementation and its user
interface and can use the library for common actions like shuffling and drawing
cards, rolling dice or moving pieces on the board without needing to care about
the communication between the players or ensuring that nobody can influence the
order of cards in a card deck or find out which cards other players have drawn.

\section*{Results}\addcontentsline{toc}{section}{Results}

The library was created according to the requirements and is ready to be used
for board and card game implementations. It can be included in web projects and
be used in compliance with modern software developement practices. The goals
have been achieved and three example implementations (two real, existing board
games of different complexity, and one demo website featuring dice and cards)
show the library's usefulness for both easy and complex board and card games. A
library manual explains how to include and use the library in own projects, a
technical documentation enlarges upon how the goals were realized, especially
the cryprographic guarantees to avoid cheating, and which factors were
considered when designing the library's programming interface. The three
implementations are operational and can be used as references or starting points
for own projects.

\section*{Outlook}\addcontentsline{toc}{section}{Outlook}

After this thesis the library will be published and can be used together with
its technical documentation, library manual, and the reference implementations
by software development teams to simplify the creation of board and card game
implementations, by board game authors to prototype and test ideas without
needing to create physical copies of all components, by board game publishers
for online game implementations (may that be as another way to earn money with
intellectual property, to let players test or play for free as a way to promote
the physical board games, or in the form of online rule learning tutorials) or
by individuals wishing to implement their favorite board games and play them
with friends without the need to develop a whole app.

There is also potential for follow-up projects: With a second library tackling
typical user interfaces (in 2D or even simulating a three-dimensional
environment) the development could be simplified even more. Also imaginable is a
plattform for creating board and card game implementations through a graphical
website through uploading existing image files, leading to evading the need of
writing programm code and allowing even people without any software development
knowledge at all to be able to bring game implementations into being. This would
drastically expand the target group and ease of use of this project.
