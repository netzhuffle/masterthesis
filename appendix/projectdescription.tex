\chapter{Project Proposal}

This chapter describes the project as it was proposed by the author for the
master thesis.

Like the original project proposal, this chapter is written in German.

\selectlanguage{ngerman}

\section{Studierender}

Jannis Grimm

\section{Semester}

HS 2016

\section{Projekttitel}

\foreignlanguage{USenglish}{Webbased API for Realization of Real-Time
Peer-to-Peer, Serverless, Installation and Plugin Free Board and Card Games With
Untrusted Third Party Opponents}

\section{Projektbetreuer}

Prof. Dr. Farhad Mehta

\section{Datum Projektstart}

16. Juni 2016

\section{Datum Projektende}

16. Dezember 2016

\section{Projektziele und Projektbeschreibung}

Es soll eine webbasierte API zur Umsetzung von Brett- und Kartenspielen
inklusive Beispielumsetzungen entwickelt und dokumentiert werden. Diese sollen
ohne Installation oder Plugin im Webbrowser nutzbar sein. Die Kommunikation
findet (mit Ausnahme des Ladens der Webapplikation und des Verbindungsaufbaus)
Peer-to-Peer ohne zentralen Server statt. Dabei muss beachtet werden, dass den
anderen Clients nicht vertraut werden kann und somit kryptografisch
sichergestellt werden, dass kein Betrug bei verdeckten Informationen oder
Zufallszahlen möglich ist.

\begin{enumerate}
  \item Die API muss die Umsetzung von verschiedenen Arten von Brett- und
  Kartenspielen ermöglichen.
  \item Zum Zeigen der Erreichung dieses Ziels muss ein nutzbares, einfaches,
  bekanntes sowie ein komplexeres, aktuelles  Brett- oder Kartenspiel umgesetzt
  werden.
  \item Die API muss mindestens folgende, typische Funktionalitäten von Brett-
  und Kartenspielen unterstützen:
  \begin{enumerate}
    \item Anzeige eines Spielbrettes
    \item Bewegen von Figuren und Markern auf dem Spielbrett
    \item Erzeugen von Zufall mittels sechsseitiger Zahlenwürfel
    \item Erzeugen von Zufall mittels gemischter Kartenstapel
    \item Karten ziehen von offenen und verdeckten Stapeln und ablegen, frei
    positioniert oder gezielt auf Ablagestapel
    \item Mischen von Ablagestapeln zur Erzeugung neuer Zugstapel
    \item Handkarten, die für andere Spieler nicht einsehbar sind
  \end{enumerate}
  \item Da Manipulation der Clients möglich ist, muss bei all jenen
  Funktionen Betrug ausgeschlossen werden:
  \begin{enumerate}
    \item Zufallszahlen müssen auf eine Art erzeugt werden, dass sie nicht von
    einzelnen Clients beeinflussbar sind.
    \item Verdeckte Karten dürfen nicht clientseitig nachträglich ausgetauscht
    werden können.
    \item Beim Ziehen von Karten eines gemischten Stapels muss ausgeschlossen
    werden können, dass der Gegner erfährt, welche Karte gezogen wurde,
    gleichzeitig jedoch ausgeschlossen werden, dass zwei Clients dieselbe
    physische Karte ziehen können oder beeinflusst werden kann, welche Karte
    gezogen wird.
  \end{enumerate}
  \item Das Verbinden von Clients auf ein gemeinsames Spiel muss über URLs
  stattfinden, die an die Mitspieler weitergegeben werden können.
  \item Sind die Clients verbunden, muss sämtliche Kommunikation Peer-to-Peer
  stattfinden, ohne einen zentralen Server zu nutzen.
  \item Die Webapplikation muss in den aktuellen Versionen von Chrome, Edge,
  Firefox und Opera funktionieren (da sich jene Browser automatisch
  aktualisieren, wird eine Unterstützung älterer Versionen dieser Browser als
  nicht notwendig angesehen).
  \item Folgende seltenere Funktionalitäten von Brett- und Kartenspielen sollten
  unterstützt werden, sofern genügend Zeit zur Realisierung ist:
  \begin{enumerate}
    \item Mehrere Spielbretter
    \item Stapeln von Markern
    \item Zweiseitige, umdrehbare Marker
    \item Würfel mit angepassten Seiten (beispielsweise Symbole)
    \item Würfel mit unterschiedlichen Seitenzahlen
  \end{enumerate}
  \item Falls es die Zeit zulässt, kann die API Möglichkeiten zur
  Automatisierung häufiger Spielaktionen oder Kontrolle der Einhaltung von
  Spielregeln möglich machen.
\end{enumerate}

Ziele 1–7 sind dabei Muss-Kriterien, Ziel 8 ist ein Soll-Kriterium und Ziel 9
ist ein Kann-Kriterium. Ausdrücklich nicht Ziel ist die Umsetzung einer
ansprechenden und optimierten grafischen Oberfläche. Da die API im Vordergrund
steht reicht eine einfache, zweckmässige Umsetzung.

\section{Projektpartner}

N/A

\section{Zu erwartende Ergebnisse}

\begin{itemize}
  \item Technischer Bericht 
  \item API-Dokumentation
  \item Quellcode der API
  \item Quellcode der Beispielumsetzungen
  \item Poster
  \item Webserver-Instanz mit der realisierten Web-Applikation
  \item CD mit allen Ergebnissen
\end{itemize}

\section{Zu erarbeitende Kompetenzen (Fach-, Methoden- und Selbstkompetenzen)}

\begin{itemize}
  \item Konzeptionierung und Erarbeiten einer zielführenden API
  \item Anwenden von Design-Patterns
  \item Selbstständiges aneignen von Kenntnissen der benötigten, modernen
  Webtechnologien
  \item Selbstständiges Erarbeiten der benötigen kryptografischen Kenntnisse
  \item Selbstständiges Erarbeiten der benötigen Kenntnisse im Bereich Verteilte
  Software Systeme
  \item Schreiben eines wissenschaftlichen Berichts
  \item Selbstständige Umsetzung eines Projekts von Idee über Planung bis
  Realisierung
  \item Strukturierte und geplante Vorgehensweise
\end{itemize}

\section{Beschreibung der Leistungsbeurteilung}

Gemäss Modulbeschreibung SWSY_MT:

\begin{itemize}
  \item Organisation / Durchführung
  \item Bericht / Sprache
  \item Analyse
  \item Entwurf
  \item Umsetzung
\end{itemize}

\section{ECTS-Credits}

27 ECTS-Credits

\selectlanguage{USenglish}
