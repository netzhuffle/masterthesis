\chapter{Requirements} \label{chap:requirements}

This chapter explains the requirements for the project.

\section{Functional Requirements}

\begin{enumerate}
  \item There must be a reference implementation of widely known board or card
  game.
  \item There must be a reference implementation of a new (i.e. published in
  the last 6 years) and more complex (i.e. a weight rating of at least 2.5 of 5
  on \cite{BGG}) board or card game.
  \item The API must support at least the following, typical functionalities of
  board and card games:
  \begin{enumerate}
    \item Showing a board
    \item Placing and moving game pieces
    \item Generate randomness through six-sided \glspl{die} numbered from 1 to 6
    \item Generate randomness through shuffled card \glspl{deck}
    \item Drawing cards from a \gls{draw pile}
    \glspl{draw pile}
    \item Placing cards, either freely or on \glspl{discard pile}
    \item Shuffling a \glspl{discard pile} to create a new \glspl{draw pile}
    \item A \gls{hand} where other players cannot see the \glspl{face}
  \end{enumerate}
  \item Connecting of clients to a joint game must happen through sharing URLs
  to other players.
  \item The following lesser common functionalities of board and card games
  should be supported if there is enough time for the implementation:
  \begin{enumerate}
    \item Multiple boards in a single game
    \item Stacking of game pieces
    \item Two-sided, reversable tokens
    \item \Glspl{die} with customizable sides (e.g. symbols)
    \item \Glspl{die} which are not 6-sided
  \end{enumerate}
  \item If time allows, the API could support the possibility for games to
  automatize common game actions or controlling that rules are not violated.
\end{enumerate}

\section{Non-Functional Requirements} \label{sec:nonfunctionalreq}

\begin{enumerate}
  \item Because manipulation of clients is possible, cheating must not be
  possible in the following functionalities:
  \begin{enumerate}
    \item \label{itm:distributedrandomness} Random numbers must be generated in
    a way not manipulable by single clients.
    \item Clients must not find out the \gls{face} of \gls{face}-down cards and
    cards on other players' \glspl{hand}.
    \item Clients must not be able to lie about \gls{face}-down cards when
    turning \gls{face}-up.
    \item \label{itm:mentalpoker} When drawing cards from a shuffled
    \gls{draw pile} it must not be possible for other clients to find out which
    card was drawn. At the same time it must not be possible for multiple
    clients to draw the same ``physical'' card. Clients must not be able to
    manipulate which card other clients draw.
  \end{enumerate}
  \item Once the clients are connected, all communication must happen
  \gls{peer-to-peer} without a central server.
  \item It must be possible to play without creating a user account first.
  \item Players must not need to install an application or plug-in first, other
  than a supported web browser.
  \item \label{itm:supportedbrowsers} The web application must be usable in
  up-to-date versions of Chrome, Firefox and Opera.\footnote{This requirement
  changed compared to the project proposal as the risk analysis in the
  inception phase showed that the APIs needed for data exchange through
  \gls{P2P} are currently not implemented in Edge, see risk
  \see{sec:riskedgedatachannels}} (As these browser automatically update to the
  newest version it is not important to support old versions.)
  \item The API must be published as open-source software.
  \item The API must be documented.
  \item The reference implementations should be published as open-source. They
  may be published under a proprietary licence if requested by the rightsholders
  or not be published at all and just be used for documentation purposes.
\end{enumerate}

\section{API Use-Case Model}

As the API is only used by board and card game implementations, there is a
single actor: The game implementation.

The Use-Case Model contains the “must” and “should” requirements but no “may”
requirements. The actions of the players inside games depend on the games and
thus are not handled here.

\subsection{Use-Case List}

\begin{enumerate}
  \item Create a game session
  \item Connect to a game session
  \item Add a board to the table
  \item Add a game piece to the table
  \item Add a \gls{draw pile} to the table
  \item Add a \gls{die} to the table
  \item Roll a \gls{die}
  \item Draw a card
  \item Move a game piece
  \item Move a card
  \item Shuffle a \gls{discard pile} to create a \gls{draw pile}
  \item Flip a game piece
\end{enumerate}
