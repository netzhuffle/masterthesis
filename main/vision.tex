\chapter{Vision}

This chapter explains the vision for the project.

\section{Current Situation}

In the growing sector of board and card games, companies started in the last few
years to letting produce their products as apps for mobile devices and
computers. This allows users to play their favorite board and card games on the
go without needing table space or over the internet with friends at other
places---as long as they have access to a plattform on which the developer
created an app for.

In contrast to native apps which have to be created for each operating system,
there is the web plattform which essentially is usable on every operating
system. This has the big advantage that apps for the web have to be created once
and can be used on every operating system---there is no need to develop and
maintain multiple apps to be able to target each user. Also, web apps can be
used without installation---this allows the userbase to grow fast as every
person who knows about the web app is immediately able to use it.

Until a few years ago, the web plattform had downsides which kept developers
from using its advantages: It was missing key functionalities (e.g.
no offline capatibilities, no 3D graphic support, no \gls{peer-to-peer}
connections, and so on). With the stabilisation of the \gls{HTML5} ecosystem
this issue changed: A lot of new standards emerged which began to close the
functionality gap between native and web apps. While browser vendors are quick
to implement the new standards in their respective browser, there was another
issue: Users where slow in adapting new versions of their web browsers. It was
not practically possible to use the new \gls{HTML5} functionallity without
excluding a big part of the users who still are using old browser version. This
now has changed in the last years: Web browsers now are automatically updating
as soon as new versions are released. This assures that most of the users are
able to use those new functionalities and newly added functionalities are soon
usable for most of the existing users.

With both of the mentioned problems solved, developers are now able to use make
use of the advantages of the web plattform. But as of today, there is no
application framework known to the author that allows creating board and card
game implementation without needing browser plugins (like Flash Player or Unity
Player) which destroy the web plattform advantages, as again installations are
needed and the plugin needs to exist for the user's operating system. Also there
is no application framework known to the author that is both open source and
generic (to support different types of board and card games). As such, there
exists a lot of unused potential for board and card games on the web plattform.

\section{Goals}

As a part of this project, a application framework for the web plattform should
be created, allowing developers to easily create board and card game
implementations and as such make use of the advantages of the web plattform for
their board and card games. This framework should make use of the new
\gls{HTML5} functionalities like \gls{peer-to-peer} connections (and thus not
requiring a server for communication), taking away the need for every developer
to reinvent the wheel and needing to develop an own solution. Through a
well-defined \gls{API} these inner workings should be taken away from the board
and card games implementation, allowing the developers to only care about their
game implementations.

The user should be able to navigate to a board or card game website with their
web browser to start a new game session. He should then be able to invite
friends by simply giving them the unique game session URL. Visiting these links,
the users automatically connect \gls{peer-to-peer} to each other, being able to
start use the chosen board or card game. By reducing the connection process for
the user to just sharing and using an URL, the biggest advantages of the web
plattform can be used---no registration, searching by usernames, or similar is
required and the players can directly jump into the game with a single click.

Together with the \gls{API}, two board or card game implementations are part of
the project. This allows testing and showcasing the \gls{API}, giving practical
examples how to use the \gls{API} and showing that the \gls{API} is generic and
applicable to real world board and card games.

There are a few challanges for this project:

\begin{description}
  \item[Generic \gls{API}:] Designing an \gls{API} which allows the realization
  of as much different games as feasible.
  \item[Trust control:] Being \gls{peer-to-peer} based, there is no trusted
  server to control or store \gls{hidden information}. Other players could use
  manipulated clients, so cheating should be prevented on the framework level.
  By example, if a user plays a \gls{face}-down card, the other user
  do not know what is on the front side of the card, but the user must not be
  able to change the card with another when revealing.
  \item[Distributed generation of randomness:] \Glspl{die} are a common element
  in board games. There must be a way to conjointly create pseudo-random numbers
  without needing to trust the other clients.
  \item[“Mental Poker Problem”:] Users must be able to draw cards from a
  shared, shuffled card pile without the other clients knowing which cards they
  are and without beeing able for multiple clients to draw the same “physical”
  card.
  \item[Reference implementations:] Finding two very different board or card
  games for the reference implementations to use as an example, to showcase all
  realized features and to show the abilities of the framework.
\end{description}

\section{Motivation}

As there is no similar project known to the author, there is a high relevancy to
practice of this project. There is a high chance the result of the master thesis
and this project are going to be used by users wanting a simple,
installation-less possibility to compete in online board and card games, as well
as by publishers wanting to offer their games for promotional proposes or even
as a paid solution.

As the author's specialisation during the master studies have been modern web
technologies, he is very interested in this project, as it allows using multiple
new web technologies that have not been around a few years ago. As a student
project research thesis during the bachelor studies he worked on a web
application with \gls{peer-to-peer} video telephony as the core feature
\cite{blaser2013practical}. It will be interesting to see how the feature and
the browser support developed since then (as the specification was back then and
still is in the Working Draft status as of June 2016 \cite{w3c2016webrtc}). He
loves learning new things and finding out how to use new technologies.

During leisure time, the author enjoys playing board and card games, has a
growing collection, visits board game fairs as an exhibitor and novelty
presenter and works on multiple board games as an editor. Thus he had the idea
for the master thesis since the beginning of his master studies and is looking
forward to finally being able to realize this project as it merges two of his
favorite subjects, has very interesting mathematic and cryptographic problems to
be solved and also contains research parts about finding the right
technologies, how to use them and solve the associated challanges.

\section{Stakeholders}

\begin{description}
  \item[Players] want to play the chosen games against each other in a
  non-complicated way without technical issues.
  \item[Developers] want to use an API to reduce the work and needed time to
  implement new board game apps by not needing to reinvent the wheel and ideally
  only needing to write their board game related program code.
  \item[Publishers] want to be able to reduce the financial costs
  involved for publishing online board game apps.
  \item[Authors] want a way to easily playtest new ideas without producing
  physical components and being able to change them as the board or card game
  processes.\footnote{Thanks to Andreas Finkernagel from Pegasus Spiele for
  pointing out the idea that this common need could also be solved by this
  project}
\end{description}

\section{Boundaries}

Creating a good looking user interface and user experience design is not scope
of this project. The complete set of requirements are defined in
\see{chap:requirements}.
