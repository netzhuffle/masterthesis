\chapter{Requirements} \label{chap:requirements}

This chapter explains the requirements for the project.

\section{Functional Requirements}

\begin{enumerate}
  \item There must be a reference implementation of a widely known board or card
  game.
  \item There must be a reference implementation of a new (i.e. published in
  the last 6 years) and more complex (i.e. a weight rating of at least 2.5 of 5
  on \cite{BGG}) board or card game.
  \item The API must support at least the following, typical functionalities of
  board and card games:
  \begin{enumerate}
    \item Showing a board
    \item Placing and moving game pieces
    \item Generate randomness through six-sided \glspl{die} numbered from 1 to 6
    \item Generate randomness through shuffled card \glspl{deck}
    \item Drawing cards from a \gls{draw pile}
    \item Placing cards, either freely or on \glspl{discard pile}
    \item Shuffling a \gls{discard pile} to create a new \gls{draw pile}
    \item A \gls{hand} where other players cannot see the \glspl{face}
  \end{enumerate}
  \item Connecting of clients to a joint game must happen through sharing URLs
  to other players.
  \item The following lesser common functionalities of board and card games
  should be supported if there is enough time for the implementation:
  \begin{enumerate}
    \item Multiple boards in a single game
    \item Stacking of game pieces
    \item Two-sided, reversable tokens
    \item \Glspl{die} with customizable sides (e.g. symbols)
    \item \Glspl{die} which are not 6-sided
  \end{enumerate}
  \item If time allows, the library could support the possibility for games to
  automatize common game actions or controlling that rules are not violated.
\end{enumerate}

\section{Non-Functional Requirements} \label{sec:nonfunctionalreq}

\begin{enumerate}
  \item Because manipulation of clients is possible, cheating must not be
  possible in the following functionalities:
  \begin{enumerate}
    \item \label{itm:distributedrandomness} Random numbers must be generated in
    a way not manipulable by single clients.
    \item Clients must not find out the \gls{face} of \gls{face}-down cards and
    cards on other players' \glspl{hand}.
    \item Clients must not be able to lie about \gls{face}-down cards when
    turning \gls{face}-up.
    \item \label{itm:mentalpoker} When drawing cards from a shuffled
    \gls{draw pile} it must not be possible for other clients to find out which
    card was drawn. At the same time it must not be possible for multiple
    clients to draw the same ``physical'' card. Clients must not be able to
    manipulate which card other clients draw.
  \end{enumerate}
  \item Once the clients are connected, all communication must happen
  \gls{peer-to-peer} without a central server.
  \item It must be possible to play without creating a user account first.
  \item Players must not need to install an application or plug-in first, other
  than a supported web browser.
  \item \label{itm:supportedbrowsers} The web application must be usable in
  up-to-date versions of Chrome, Firefox and Opera.\footnote{This requirement
  changed compared to the project proposal as the risk analysis in the
  inception phase showed that the APIs needed for data exchange through
  \gls{P2P} are currently not implemented in Edge, see risk
  \see{sec:riskedgedatachannels}} (As these browsers automatically update to the
  newest version it is not important to support old versions.)
  \item The API must be published as open-source software.
  \item The API must be documented.
  \item The reference implementations should be published as open-source. They
  may be published under a proprietary licence if requested by the rightsholders
  or not be published at all and just be used for documentation purposes.
\end{enumerate}

\section{API Use-Case Model}

As the API is only used by board and card game implementations, there is a
single actor: The game implementation.

The Use-Case Model contains the ``must'' and ``should'' requirements but no ``may''
requirements. The actions of the players during games depend on the games and
thus are not handled here.

\subsection{Use-Case List}

As the API is used by game implementations and not directly by players, the
``Use Cases'' are not typical user-based Use Cases (e.g. shorter and only a
few steps). From the user view, there would be a single use case ``Play game''.

\begin{enumerate}
  \item Create a game session
  \item Connect to a game session
  \item Start a game session
  \item Add a board to the table
  \item Add a game piece to the table
  \item Add a \gls{draw pile} to the table
  \item Add a \gls{die} to the table
  \item Roll a \gls{die}
  \item Draw a card
  \item Move a game piece
  \item Move a card
  \item Shuffle a \gls{discard pile} to create a \gls{draw pile}
  \item Flip a card
  \item Flip a game piece
\end{enumerate}

The most important five Use Cases are explained in the fully dressed format
below.

\subsection{Create a Game Session}

\begin{description}
  \item[Brief Description:] The user creates a game session for other players
  to join.
  \item[Stakeholders and Interests:] Players want to create a new game session
  and an easy way to invite other players. Developers want to create and setup a
  game session to connect the players over the network without needing to
  implement the connection and communication logic by themselves. System
  administrators want to have a low server load even with many players joining.
  \item[Preconditions:] None.
  \item[Success Guarantee:] The client has created and joined the new game
  session and is able to send and receive events to and from joining players.
  The user got an URL to share with other players through which they can join
  the game.
  \item[Main Success Scenario:] \hfill
  \begin{enumerate}
    \item The game implementation tells the framework that a user wants to
    create a new game session.
    \item The framework creates a new game identifier.
    \item The framework readies the new session for other players to join.
    \item The game implementation can tell the user an URL to share with the
    other players.
  \end{enumerate}
  \item[Extensions (List):] \hfill
  \begin{starenum}
    \item A JavaScript error occures.
    \item The used client does not support a needed API.
  \end{starenum}
  \item[Frequency of Occurrence:] Once per game session.
\end{description}

\subsection{Connect to a Game Session}

\begin{description}
  \item[Brief Description:] The user connects to a game session to enter the
  game as a player.
  \item[Stakeholders and Interests:] Players want to join a game without
  technical issues. Developers want to connect the player to the others over the
  network without needing to implement the connection and communication logic by
  themselves. System administrators want to have a low server load even with
  many players joining.
  \item[Preconditions:] A game session is created and the joining client knows
  its identification.
  \item[Success Guarantee:] The client has joined the game sessions. The other
  clients in the game session know about the new client. The new client is able
  to receive events from the other clients and vice versa.
  \item[Main Success Scenario:] \hfill
  \begin{enumerate}
    \item The game implementation tells the framework the game session
    identifier.
    \item \label{itm:connectgamesession_lookup} The framework looks up on a
    web server who the other clients in the game session are.
    \item \label{itm:connectgamesession_connect} The framework connects the new
    client to the other clients.
  \end{enumerate}
  \item[Extensions (List):] \hfill
  \begin{starenum}
    \item A JavaScript error occures.
    \item The used client does not support a needed API.
  \end{starenum}
  \begin{refenum}{itm:connectgamesession_lookup}
    \item The session identifier does not exist.
  \end{refenum}
  \begin{refenum}{itm:connectgamesession_connect}
    \item The connection to a single client fails.
    \item The connections to all other clients fail.
  \end{refenum}
  \item[Frquency of Occurrence:] A few times per game session.
  \item[Miscellaneous:] \hfill
  \begin{itemize}
    \item Spectator mode for clients to just receive events but not allowed to
    manipulate the game area?
  \end{itemize}
\end{description}

\subsection{Start a Game Session}

\begin{description}
  \item[Brief Description:] The user starts the game in order that the game
  area is setup and all players are able to manipulate the game area and see
  the actions of other players.
  \item[Stakeholders and Interests:] Players want to start the game without
  technical issues. Developers want to get the possibility to manipulate the
  game area for all clients and receive events from other clients without
  needing to implement the connection and communication logic by themselves.
  System administrators want to have a low server load even when many players
  have joined.
  \item[Preconditions:] All players have joined the session and are connected
  to each other.
  \item[Success Guarantee:] The clients can manipulate the game area in a way
  that the other clients are notified of the changes.
  \item[Main Success Scenario:] \hfill
  \begin{enumerate}
    \item The game implementation tells the framework that all players have
    joined.
    \item The framework notifies all clients about the game start.
    \item The framework gives the game implementation on each client a way to
    manipulate the game area in a way that all clients know about the changes
    and to receive events from other clients.
  \end{enumerate}
  \item[Extensions (List):] \hfill
  \begin{starenum}
    \item A JavaScript error occures.
    \item A client left in the meantime.
  \end{starenum}
  \item[Frquency of Occurrence:] Once per game session.
  \item[Miscellaneous:] \hfill
\end{description}

\subsection{Add a Draw Pile to the Table}

\begin{description}
  \item[Brief Description:] The user wants to put a \gls{face}-down and shuffled
  \gls{draw pile} in the game area so all clients are able to draw cards.
  \item[Stakeholders and Interests:] Players want to have access to the
  \gls{draw pile} to be able to draw cards. They want to have the \gls{draw
  pile} shuffled in a way that other players cannot know or influence the order
  of cards. Developers want to setup the game area without needing to solve and
  implement the logic by themselves. System administrators do not want system
  load on their server through this Use Case.
  \item[Preconditions:] The client is connected to a game session. All clients
  know the cards in the \gls{deck}.
  \item[Success Guarantee:] All connected clients know about the new \gls{draw
  pile}. No client knows about the order of cards or is able to influence it.
  \item[Main Success Scenario:] \hfill
  \begin{enumerate}
    \item \label{itm:adddrawpile_telldeck} The game implementation tells the
    framework which \gls{deck} should be shuffled.
    \item The framework shuffles the cards with the help of all clients to
    create a \gls{draw pile}.
    \item The framework tells all clients about the new \gls{draw pile}.
  \end{enumerate}
  \item[Extensions (List):] \hfill
  \begin{starenum}
    \item A JavaScript error occures.
    \item Communication to a client breaks.
    \item Communication to all other clients breaks.
  \end{starenum}
  \begin{refenum}{itm:adddrawpile_telldeck}
    \item The \gls{deck} does not exist.
  \end{refenum}
  \item[Frquency of Occurrence:] Once or few times per game session.
  \item[Miscellaneous:] \hfill
  \begin{itemize}
    \item What about \gls{face}-up \glspl{draw pile}?
  \end{itemize}
\end{description}

\subsection{Draw a Card}

\begin{description}
  \item[Brief Description:] The user wants to draw a card from a \gls{draw
  pile} to his \gls{hand}.
  \item[Stakeholders and Interests:] Players want to draw a card from an
  \gls{draw pile} to use it later in the game. They do not want other players
  to be able to know or influence which card is drawn or to draw the same card.
  Developers do not want multiple players to be able to draw the same card and
  they do not want to solve and implement the logic by themselves. System
  administrators do not want system load on their server through this Use Case.
  \item[Preconditions:] The client is connected to a game session. All clients
  know about the \gls{draw pile}.
  \item[Success Guarantee:] All connected clients know that the player has drawn
  a card to his hand. The player knows which card he drew. No other client knows
  which card it is. The card is not one another player already drew.
  \item[Main Success Scenario:] \hfill
  \begin{enumerate}
    \item \label{itm:drawcard_telldrawpile} The game implementation tells the
    framework from which \gls{draw pile} the player wants to draw.
    \item The framework finds out with the help of all clients which card it is.
    \item The framework tells all clients that the player drew a card.
    \item The framework tells the player's client which card it is.
  \end{enumerate}
  \item[Extensions (List):] \hfill
  \begin{starenum}
    \item A JavaScript error occures.
    \item Communication to a client breaks.
    \item Communication to all other clients breaks.
  \end{starenum}
  \begin{refenum}{itm:drawcard_telldrawpile}
    \item The \gls{draw pile} does not exist.
    \item The \gls{draw pile} is empty.
  \end{refenum}
  \item[Frquency of Occurrence:] Multiple times per game session.
  \item[Miscellaneous:] \hfill
  \begin{itemize}
    \item What about \gls{face}-up \glspl{draw pile}?
  \end{itemize}
\end{description}
